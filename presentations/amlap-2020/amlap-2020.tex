\documentclass{slides}
\usepackage[british]{babel}
\usepackage[utf8]{inputenc}
\usepackage[T1]{fontenc}
\usepackage{csquotes}
\usepackage[style=apa, sortcites=true, sorting=nyt,backend=biber]{biblatex}
\DeclareLanguageMapping{british}{british-apa}
\addbibresource{references.bib}
\usepackage{framed, color}
\definecolor{shadecolor}{rgb}{1,0.8,0.3}

\graphicspath{{./gfx/}}
% Frame number
\setbeamertemplate{footline}[frame number]


\title[]{ No scope for planning -- language pre-planning as mixture process }

\author{\small Jens Roeser \\ $\phantom{foo}$ \\ Mark Torrance $\phantom{foo}$ Mark Andrews $\phantom{foo}$ Thom Baguley \\$\phantom{foo}$ \\ Nottingham Trent University, UK \\ \url{jens.roeser@ntu.ac.uk} }

\institute{26th AMLaP, University of Potsdam}
\date{Sept 3, 2020}

\begin{document}\frame{\titlepage}


% 1. Planning scope variability: under pressure people dedicate less time to planning
% 2. Planning scope effect: certain syntactic configuration require more preplanning (regardless of syn adjacency)

% 2 is evidence for syntactic configuration guiding advance planning
% 1 is evidence against this


% flexibility comes from whether or not lexical information within the syntactic scope is activated
% link to coefficients: 1 sec is too long for only first picture (scene gist, message)

	\input{slides/intro}
	
\begin{frame}{Research focus}

\begin{itemize}
	\item Direct comparison of two hypothesis.
	\item[i.] \uncover<1->{Phrase scope obligated by the production system, leading to a systematic slowdown for conjoined NPs.}
	\item[ii.] \uncover<1->{Preplanning beyond the first noun is more likely in conjoined NPs but not obligated by the production system.} 


\end{itemize}

\end{frame}


\begin{frame}{Pooled re-analysis}


\begin{itemize}
	\item Stimulus-to-onset latencies
	\item[a.] \textbf{Conjoined NPs:} \textit{The boy and the dog moved above the kite.}
	\item[b.] \textbf{Simple NPs:} \textit{The boy moved above the dog and the kite.}	
\end{itemize}


\begin{itemize}
	\item \textcite{hardy2019age}: 90 ppts; 36 items
	\item \textcite{hardy2020healthy}: 105 ppts; 80 items
	\item \textcite{martin2010planning}: 3$\times$12 ppts; 2$\times$48 and 1$\times$64 items
	\item \textcite{roeser2019advance}: 3$\times$32 ppts; 96 items 
\end{itemize}


\end{frame}


\begin{frame}{Data overview}
	\begin{flushright}
		\includegraphics[scale=.5]{metaraw.pdf}
	\end{flushright}
\end{frame}






\begin{frame}[fragile]{Model overview}

\begin{minipage}[t]{.53\textwidth}
\uncover<-2>{
\begin{enumerate}
		\item Null LMM
		\item \textbf{LMM (NP effect)}
		\item LMM (unequal variance)
		\item Null mixture model
		\item \textbf{Mixture model}
\end{enumerate}
\vfill
\begin{itemize}
 	\item Stan code based on \textcite{sorensen2016bayesian} and \textcite{vasishth2017}; also \textcite{vasishth2017feature}.
\end{itemize}

}
\end{minipage}
\hfill
\begin{minipage}[t]{.45\textwidth}
\uncover<2->{
\begin{itemize}
	\item $LogNormal$ distribution with mean $\mu$ and error variance $\sigma_e^2$
	\item Random intercepts 
	\begin{itemize}
		\item participants: $u_i \sim Normal(0, \sigma_u^2)$
		\item items: $w_j \sim Normal(0, \sigma_w^2)$
	\end{itemize}
	\item Weakly informative priors \parencite{mcelreath2016statistical}
\end{itemize}
}
\end{minipage}
	
\end{frame}

\begin{frame}[fragile]{Null model (null hypothesis)}
	
		\begin{equation*}
			\begin{aligned}	
				y_{ij} \sim LogNormal(\mu_{ij}, \sigma_e^2) \\
				\mu_{ij} = \alpha + u_i + w_j\\
			\end{aligned}
		\end{equation*}
\end{frame}


\begin{frame}[fragile]{Meta null model (null hypothesis)}
	
		\begin{equation*}
			\begin{aligned}	
				y_{ijk} \sim LogNormal(\mu_{ijk}, \sigma_{e_k}^2) \\
				\mu_{ijk} = \alpha_k + u_i + w_j\\
				\alpha_k = \alpha_{\mu} + \alpha_{\tau} \cdot \alpha_{\eta_k}\\
			\end{aligned}
		\end{equation*}
		\begin{small}	
			\begin{itemize}
				\item For $k = 1, \dots, K$ where $K$ is the number of studies.
				\item $\alpha_k$ is the latency coefficient for the $k$th study.
				\item $\alpha_{\mu}$ is the pooled latency coefficient.
				\item Non-centred parametrisation for $\alpha_k$ \parencite{gelman2014}.
			\end{itemize}
		\end{small}
\end{frame}


\begin{frame}[fragile]{Meta LMM (standard analysis)}
			
	\begin{equation*}
		\begin{aligned}	
			y_{ijk} \sim LogNormal(\mu_{ijk}, \sigma_{e_k}^2)\\
			\mu_{ijk} = \alpha_k + \beta_k \cdot x_{[0,1]} + u_i + w_j\\
			\alpha_k = \alpha_{\mu} + \alpha_{\tau} \cdot \alpha_{\eta_k}\\
			\beta_k = \beta_{\mu} + \beta_{\tau} \cdot \beta_{\eta_k}\\
		\end{aligned}	
	\end{equation*}		
	\begin{small}	
		\begin{itemize}
			\item $x=0$ for simple NPs; $x=1$ for conjoined NPs.
			\item $\beta_k$ is the latency change for conjoined NPs for the $k$th study.
			\item $\beta_{\mu}$ is the pooled latency change for conjoined NPs.
		\end{itemize}
	\end{small}
	
\end{frame}


\begin{frame}[fragile]{Mixture model (alternative hypothesis)}
	
	\begin{equation*}
		\begin{aligned}
		y_{ij} \sim \theta_{NP} \cdot LogNormal(\mu_{ij} + \delta, \sigma_{e'}^2) + \\
			(1 - \theta_{NP}) \cdot LogNormal(\mu_{ij}, \sigma_{e}^2) \\
			\mu_{ij} = \alpha + u_i + w_j\\
			\text{constraint: }\delta>0\\
			\sigma_{e'}^2 > \sigma_{e}^2 
		\end{aligned}
	\end{equation*}

\uncover<2>{
	\begin{small}
		\begin{itemize}
%			\item $\mu_{ij}$ defined as before.
			\item Probability of long latencies $\theta$ by NP type.
			\item $\mu$ and $\sigma^2$ constant across NP type.
		\end{itemize}
	\end{small}		
}
	
\end{frame}

\begin{frame}[fragile]{Meta mixture model (alternative hypothesis)}

	\begin{equation*}
		\begin{aligned}
		y_{ijk} \sim \theta_{{NP}_k} \cdot LogNormal(\mu_{ijk} + \delta_k, \sigma_{e'_k}^2) + \\
			(1 - \theta_{{NP}_k}) \cdot LogNormal(\mu_{ijk}, \sigma_{e_k}^2) \\
			\mu_{ijk} = \alpha_k + u_i + w_j\\
			\theta_{{NP}_k} = Logit^{-1}(\phi_{{NP}_k})\\
%			\theta_{{NP}} = Logit^{-1}(\phi_{_{\mu_{NP}}})\\		
			\phi_{{NP}_k} \sim Normal(\phi_{\mu_{NP}}, \phi_{\tau}^2)\\
			\delta_k \sim Normal(\delta_{\mu}, \delta_{\tau}^2)\\
			\text{constraint: }\delta_k>0\\
		\end{aligned}
	\end{equation*}
	
	\begin{small}	
		\begin{itemize}
			\item $\alpha_{k}$, $\sigma_{e'_k}^2$, $\sigma_{e_k}^2$ defined as before.
			\item Pooled coefficient for parameters: $\theta_{NP}$, $\delta$
%			\item Inverse-logit for continuous prior on mixing proportion.
		\end{itemize}
	\end{small}
	
\end{frame}


\begin{frame}[fragile]{Meta LMM (unequal variance)}
	\begin{equation*}
		\begin{aligned}
		y_{ijk} \sim
			\begin{dcases*} 
				LogNormal(\mu_{ijk}, \sigma_{e_k}^2), &  if NP$_{ijk}$ = simple\\
				LogNormal(\mu_{ijk} + \beta_k, \sigma_{e'_k}^2) & else if NP$_{ijk}$ = conjoined\\
			\end{dcases*}\\
		\mu_{ijk} = \alpha_k + u_i + w_j\\
		\alpha_k = \alpha_{\mu} + \alpha_{\tau} \cdot \alpha_{\eta_k}\\
		\beta_k = \beta_{\mu} + \beta_{\tau} \cdot \beta_{\eta_k}\\
		\text{constraint: }\sigma_{e_k}^2>0\\
		\sigma_{e'_k}^2 > \sigma_{e_k}^2
		\end{aligned}
	\end{equation*}
	
	
\end{frame}


\begin{comment}
\begin{frame}{Predictions}

\begin{itemize}
	\uncover<-1>	{\item Standard analysis (LMM): Conjoined NPs cause a systematic slowdown in onset latencies, implying that phrase syntax is obligated by the production system.}
	\uncover<2>	{\item Alternative analysis (MoG): Conjoined NPs show a larger probability for longer onset latencies which, however, remain the minority; hence, preplanning syntax is not obligated by the production system.}
\end{itemize}

\end{frame}
\end{comment}




	\input{slides/results-meta}
	\begin{frame}{Summary}

\begin{itemize}
	\item The frequently observed slowdown for conjoined NPs is better explained by a larger, yet relatively small, probability of long latencies.
	\item Different pattern for \textcite{hardy2019age,hardy2020healthy} compared to  \textcite{martin2010planning} and \textcite{roeser2019advance}; possibly because of data trimming threshold.
	\item Most studies in our pool included other manipulations.
\end{itemize}
\end{frame}

\begin{frame}{Follow-up experiments}

\setlength{\leftmargini}{0.5cm}
\setlength{\leftmarginii}{0.5cm}
\begin{itemize}
	\item[] \textbf{Experiment 1:} 
	\begin{itemize}
		\item Reproduce analysis after \dots
		\item[i.] Reducing the manipulation to simple and conjoined NPs.
		\item[ii.] Controlling image names
	\end{itemize}
	\item[] \textbf{Experiment 2:} 
	\begin{itemize}
		\item Assess the impact of the visual manipulation independently of utterance sytnax.
		\item Elicit name lists instead of sentences \parencite[as in][]{martin2010planning}
	\end{itemize}
\end{itemize}

\end{frame}
	\input{slides/method-exps}
	

\begin{frame}{NP-type effect (LMM)}

\begin{tikzpicture}
    \draw (0, 0) node[inner sep=0,anchor=west] {\includegraphics[scale=.5]{NPexp.pdf}};
    \draw (2.75, 2.9) node[font=\tiny,anchor=west] {LMM-1 -- LMM-0: $\updel\widehat{elpd}=$-6 (3)};
    \draw (2.75, 2.5) node[font=\tiny,anchor=west] {LMM-1: $\widehat{elpd}=$-22,125 (78)};
    \draw (10.75, .35) node[font=\tiny,anchor=east] {LMM-1 -- LMM-0: $\updel\widehat{elpd}=$0 (1)};
    \draw (10.75, -.05) node[font=\tiny,anchor=east] {LMM-1: $\widehat{elpd}=$-14,318 (69)};
\end{tikzpicture}

\end{frame}

\begin{frame}{Probability of long latencies (MoG)} 

\begin{tikzpicture}
    \draw (0, 0) node[inner sep=0,anchor=west] {\includegraphics[scale=.5]{NPexpmog.pdf}};
    \draw (10.75, 2.95) node[font=\tiny,anchor=east] {MoG-1 -- LMM-1: $\updel\widehat{elpd}=$-401 (36)};
    \draw (10.75, 2.55) node[font=\tiny,anchor=east] {MoG-1: $\widehat{elpd}=$-21,725 (65)};
    \draw (10.75, .45) node[font=\tiny,anchor=east] {MoG-1 -- LMM-1: $\updel\widehat{elpd}$=-305 (35)};
    \draw (10.75, .05) node[font=\tiny,anchor=east] {MoG-1: $\widehat{elpd}$=-14,013 (54)};
\end{tikzpicture}

\end{frame}

	% summary of claim

\begin{frame}{Summary}
	\begin{itemize}
		\item No evidence for phrase-as-planning-unit hypothesis: NP syntax isn't obligated by production system.
		\item Instead, the slowdown for conjoined NPs \parencite[as in][]{martin2010planning,smi99} is better explained by a larger probability of long latencies which, however, remained in a minority.
		\item Syntax in language production must result from a non-deterministic planning mechanism.		
	\end{itemize}
\end{frame}

\begin{comment}
\begin{frame}{Alternative explanations}
	\begin{itemize}
%		\item Results are not a by-product of the visual manipulation. %observed in a subset of trials origins during visual rather than grammatical encoding. 
%		\item Both a relational and non-relational route are available during high level encoding \parencite{kuchinsky2011reversing,konopka2014priming}. 
		\item[i.] Lexical preplanning: avoidance of intra-sentential pausing.
		\item[ii.] Syntactic correction of incorrectly activated NP syntax.
		\item[iii.] Both: use of syntactic route, instead of a lexical route, leads to a slowdown in conjoined NPs but not in simple NPs. 
	\end{itemize}
\end{frame}
\end{comment}

	\input{slides/thanks}
	\input{references}	
	\input{slides/equations}
	\input{slides/meta-mix_comps}
	\input{slides/exps-mix_comps}
	\begin{frame}{Model comparisons}
\begin{scriptsize}
% latex table generated in R 4.0.2 by xtable 1.8-4 package
% Wed Aug 19 16:26:17 2020
\begin{table}[ht]
\centering
	\caption{\scriptsize{Predictive performance estimated as the \textit{expected log pointwise predictive density} ($\widehat{elpd}$). Models are ordered by predictive performance (model with highest predictive performance in top row)}. Standard error in parentheses.}
\begin{tabular}{rlrr}
  \toprule
 & Model & $\updel\widehat{elpd}$ & $\widehat{elpd}$ \\ 
  \midrule
  \rowcolor{yellow!40!white}Expeirment 1 & MoG-1 & -- & -21,724 (65) \\ 
  & MoG-0 & -1 (2) & -21,725 (65) \\ 
  \rowcolor{yellow!40!white}& LMM-1 & -401 (36) & -22,125 (78) \\ 
  & LMM-2 & -402 (36) & -22,126 (78) \\ 
  & LMM-0 & -406 (36) & -22,131 (78) \\ 
   \midrule
    Expeirment 2 & MoG-0 & -- & -14,012 (54) \\ 
   \rowcolor{yellow!40!white}& MoG-1 & -1 (1) & -14,013 (54) \\ 
   & LMM-0 & -306 (35) & -14,318 (69) \\ 
   \rowcolor{yellow!40!white}& LMM-1 & -306 (35) & -14,318 (69) \\ 
   & LMM-2 & -308 (35) & -14,319 (69) \\ 

		\bottomrule
		\end{tabular}\\
		\textit{Note.} LMM = Linear mixed effects model; MoG = Mixture of Gaussians 
		\end{table}
\end{scriptsize}
\end{frame}
	
\end{document}


