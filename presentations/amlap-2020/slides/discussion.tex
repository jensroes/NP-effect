% summary of claim

\begin{frame}{Summary}
	\begin{itemize}
		\item Evidence against the phrase-as-planning-unit hypothesis: preplanning NP syntax is not obligated by the language production system.
		\item Instead, the slowdown for conjoined NPs \parencite[as in][]{martin2010planning,smi99} is better explained by a larger tendency to exhibit long onset latencies.
	\end{itemize}
\end{frame}

\begin{frame}{Alternative explanations}
	\begin{itemize}
		\item Preplanning scope is in-line with non-deterministic theories of language production.		
		\item Together with Exp.~2, results suggest that the slowdown observed in a subset of trials origins during visual rather than grammatical encoding. 
		\item Both a relational and non-relational route are available during high level encoding \parencite{kuchinsky2011reversing,konopka2014priming}. 
%		\item[i.] Lexical: avoidance of intra-sentential pausing.
%		\item[ii.] Syntactic: correction of incorrectly activated NP syntax.
%		\item[iii.] Both: use of syntactic route, instead of a lexical route, leads to a slowdown in conjoined NPs but not in simple NPs. 
	\end{itemize}
\end{frame}

