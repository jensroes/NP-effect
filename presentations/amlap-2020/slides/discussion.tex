% summary of claim

\begin{frame}{Summary}
	\begin{itemize}
		\item No evidence for phrase-as-planning-unit hypothesis: NP syntax isn't obligated by production system.
		\item Instead, the slowdown for conjoined NPs \parencite[as in][]{martin2010planning,smi99} is better explained by a larger probability of long latencies which, however, remained in a minority.
		\item Syntax in language production must result from a non-deterministic planning mechanism.		
	\end{itemize}
\end{frame}

\begin{comment}
\begin{frame}{Alternative explanations}
	\begin{itemize}
%		\item Results are not a by-product of the visual manipulation. %observed in a subset of trials origins during visual rather than grammatical encoding. 
%		\item Both a relational and non-relational route are available during high level encoding \parencite{kuchinsky2011reversing,konopka2014priming}. 
		\item[i.] Lexical preplanning: avoidance of intra-sentential pausing.
		\item[ii.] Syntactic correction of incorrectly activated NP syntax.
		\item[iii.] Both: use of syntactic route, instead of a lexical route, leads to a slowdown in conjoined NPs but not in simple NPs. 
	\end{itemize}
\end{frame}
\end{comment}
